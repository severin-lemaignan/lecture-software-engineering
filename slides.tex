%!TEX program = xelatex

\documentclass[compress]{beamer}
%--------------------------------------------------------------------------
% Common packages
%--------------------------------------------------------------------------
\usepackage[english]{babel}
\usepackage{pgfpages} % required for notes on second screen
\usepackage{graphicx}
\usepackage{subfigure}
\usepackage{multicol}
\usepackage{multirow}
\def\block(#1,#2)#3{\multicolumn{#2}{c}{\multirow{#1}{*}{$ #3 $}}}

\usepackage{fontspec}

\usepackage{tabularx,ragged2e}
\usepackage{booktabs}

\usepackage{setspace}

\usepackage{gitdags}
\usepackage[normalem]{ulem} % strikeout
%--------------------------------------------------------------------------
% Load theme
%--------------------------------------------------------------------------
\usetheme{hri}

\usepackage{tikz}
\usetikzlibrary{intersections,arrows,shapes,calc,mindmap,backgrounds,positioning,svg.path}

\graphicspath{{figs/}}

\setbeamercolor{highlightCol}{bg=hriSec3,fg=white}
\newcommand{\highlight}[1]{%
    \vspace{1em}%
    \begin{beamercolorbox}[wd=\linewidth,ht=2ex,dp=0.7ex]{highlightCol}%
    \centering #1%
    \end{beamercolorbox}%
    \vspace{1em}%
}%

\tikzset{temporal/.code args={<#1>#2#3#4}{%
          \temporal<#1>{\pgfkeysalso{#2}}{\pgfkeysalso{#3}}{\pgfkeysalso{#4}} % \pgfkeysalso doesn't change the path
}}


%--------------------------------------------------------------------------
% General presentation settings
%--------------------------------------------------------------------------
\title{\bf git}
\subtitle{the basics}
\date{24 Jan. 2017}
\author{Séverin Lemaignan}
\institute{Centre for Robotics \& Neural
Systems\\{\bf Plymouth University}}

%--------------------------------------------------------------------------
% Notes settings
%--------------------------------------------------------------------------
%\setbeameroption{show notes on second screen}
%\setbeameroption{hide notes}

\begin{document}

\licenseframe{https://github.com/severin-lemaignan/git-presentation/}
\maketitle

%%%%%%%%%%%%%%%%%%%%%%%%%%%%%%%%%%%%%%%%%%%%%%%%%%%%%%%%%%%%%%%%%%%%%%%%%%%%%%%

\section{First things first: Why bother?}

\imageframe{zipfiles.pdf}
\imageframe{dropbox_zipfiles.pdf}

{
    \fullbackground{../figs/facebook-wall.png}
    \begin{frame}[plain]{}
        \only<2>{
            \begin{center}
                \includegraphics[width=\linewidth]{devil}
            \end{center}
        }
    \end{frame}
}

\begin{frame}{Why code versioning?}

    \begin{itemize}
        \item The history of your development/document
        \item Compare the current code with an older version
        \item Roll-back to previous versions
        \item Experiment without losing anything
        \item Trace who did what (at the level of the line of code)
        \item Annotate your workflow (important milestones, etc)
        \item Avoid catastrophes!
    \end{itemize}
\end{frame}

\section{Code versioning}

\begin{frame}{Atomic commits}

    The single most important concept (because it requires to think about
    development in terms of {\bf functional units}):

    \highlight{Atomic commit}

    \only<1>{
        A (typically small) commit that represent a {\bf single, coherent \&
    complete} functional change.
    }
    \only<2>{
    \begin{itemize}
        \item Easy to understand the change
        \item Debugging made easy (\texttt{git bisect})
        \item Collaboration made easy (less, smaller conflict)
        \item Easy to write a useful commit message
    \end{itemize}
    }

\end{frame}

\begin{frame}[fragile]{}

\centering

\begin{tikzpicture}[
    >=latex,
    every edge/.style={draw,thick,hriSec3},
    file/.style={align=left, inner sep=1pt,font=\scriptsize\tt}
]

\node at (-2, 1.5)[file,temporal=<4>{red}{black}{white}] (file1) {\sout{main.cpp}};
\node [file,temporal=<4>{red}{black}{white},below=0.0 of file1.south east,anchor=north east] (file2) {src/main.cpp};
\node [file,temporal=<2>{red}{black}{white},below=0.0 of file2.south east, anchor=north east] (file3) {src/position.cpp};
\node [file,temporal=<2>{red}{black}{white},below=0.0 of file3.south east,anchor=north east] (file4) {src/position.hpp};
\node [file,temporal=<3>{red}{black}{white},below=0.0 of file4.south east,anchor=north east] (file5) {share/model.csv};


    \onslide<2>{
        \gitDAG[grow right sep = 2em]{
            A;
        };

        \gitbranch
        {master}
        {above=of A}
        {A}

        \gitHEAD
        {above=of master}
        {master}


    \path[draw] (file3.east) edge [->, bend left] (A);
    \path[draw] (file4.east) edge [->, bend left] (A);

    }
    \onslide<3>{
        \gitDAG[grow right sep = 2em]{
            A -- B;
        };

        \gitbranch
        {master}
        {above=of B}
        {B}

        \gitHEAD
        {above=of master}
        {master}

        \path[draw] (file5.east) edge [->, bend left,hriSec2] (B);
    }

    \onslide<4->{
        \gitDAG[grow right sep = 2em]{
            A -- B -- C;
        };

        \gitbranch
        {master}     % node name and text 
        {above=of C} % node placement
        {C}          % target

        \gitHEAD
        {above=of master} % node placement
        {master}          % target

        \path[draw] (file1.east) edge [->, bend left,hriSec1Comp] (C);
        \path[draw] (file2.east) edge [->, bend left,hriSec1Comp] (C);
    }

\end{tikzpicture}

\vspace{3em}
\small
\only<2>{
\texttt{git add src/position.*}\\
\texttt{git commit -m"Fix computation of position (float->double)"}
}
\only<3>{
\texttt{git add share/model.csv}\\
\texttt{git commit -m"Re-trained model with 52 more participants"}
}
\uncover<4>{
\texttt{git add src/main.*}\\
\texttt{git commit -m"Move main.cpp to src/"}
}

\end{frame}

\begin{frame}[fragile]{Log}
\begin{shcode}
$ git log
commit fa009cd7fca05b0b61170b20cf76a5f72b8843c2
Author: Severin Lemaignan <severin.lemaignan@plymouth.ac.uk>
Date:   Wed Feb 10 16:48:22 2016 +0000

    Move main.cpp to src/

commit aff81119459d9193c09effef1c150c4f7eac08dc
Author: Severin Lemaignan <severin.lemaignan@plymouth.ac.uk>
Date:   Wed Feb 10 16:48:02 2016 +0000

    Re-trained model with 52 more participants

commit 4113b9b6e6bbc8de532ad90153e0059cb5819de7
Author: Severin Lemaignan <severin.lemaignan@plymouth.ac.uk>
Date:   Wed Feb 10 16:47:46 2016 +0000

    Fix computation of position (float->double)
\end{shcode}


\end{frame}

\begin{frame}{}

\centering

\begin{tikzpicture}
        \gitDAG[grow right sep = 2em]{
            4113b9 -- aff811 -- fa009c;
        };

        \gitbranch
        {master}     % node name and text 
        {above=of fa009c} % node placement
        {fa009c}          % target

        \gitHEAD
        {above=of master} % node placement
        {master}          % target

\end{tikzpicture}

\end{frame}

\begin{frame}{The staging area}
    \centering
    \only<1>{
        But why do we have to manually tell Git what files to add or remove?
    }
    \only<2>{
        No ``commit all changes'' by default {\tiny (well, you
        can, actually...)}\\
        $\Rightarrow$ Help thinking in terms of atomic commits!
    }
    \only<3->{
        Preparing a commit consists in filling the {\bf staging area} (or
        {\bf index}) with the list of changes:

        \begin{tikzpicture}[
                >=latex,
                every edge/.style={draw,thick,hriSec3},
                file/.style={align=left, inner sep=1pt,font=\scriptsize\tt}
            ]

            \node at (-3.5, 2)[file,red] (file1) {\sout{main.cpp}};
            \node [file,red,below=0.0 of file1.south east,anchor=north east] (file2) {src/main.cpp};
            \node [file,temporal=<3>{red}{black}{white},below=0.0 of file2.south east, anchor=north east] (file3) {src/position.cpp};
            \node [file,temporal=<3>{red}{black}{white},below=0.0 of file3.south east,anchor=north east] (file4) {src/position.hpp};
            \node [file,red,below=0.0 of file4.south east,anchor=north east] (file5) {share/model.csv};


            \node[gitSA,] (stagingarea) at (-2.5,3) {staging area};

            \onslide<3>{
                \gitDAG[grow right sep = 2em]{
                    154ce2 -- f327ba;
                };

                \gitbranch
                {master}     % node name and text 
                {above=of f327ba} % node placement
                {f327ba}          % target

                \gitHEAD
                {above=of master} % node placement
                {master}          % target



                \path[draw] (file3.east) edge [->, bend right,hriSec1Comp] (stagingarea);
                \path[draw] (file4.east) edge [->, bend right,hriSec1Comp] (stagingarea);

            }
            \onslide<4>{
                \gitDAG[grow right sep = 2em]{
                    154ce2 -- f327ba -- 4113b9;
                };

                \gitbranch
                {master}     % node name and text 
                {above=of 4113b9} % node placement
                {4113b9}          % target

                \gitHEAD
                {above=of master} % node placement
                {master}          % target

                \node[highlighted commit] at (4113b9) {\phantom{4113b9}};
                \draw[resetarrows] ([xshift=-1em,yshift=-1em]stagingarea.east) to[bend
                left] (4113b9.north west);

            }



        \end{tikzpicture}


        \only<3>{
            \texttt{git add}\\
            \texttt{git rm}\\
            \texttt{git add -p}\\
            ...\\
        }
        \only<4>{
            ~\\
            \texttt{git commit}\\
            ~\\
            ~\\
        }


    }
\end{frame}

\begin{frame}[fragile]{To summarize...}

The first time...
\begin{shcode}
$ mkdir my_repo && cd my_repo
$ git init
\end{shcode}
Then...
\begin{shcode}
# make some changes...
$ git add <files>
$ git commit -m"<commit message>"
# make some changes...
$ git add <files>
$ git commit -m"<other commit message>"
# That's it!
\end{shcode}


\end{frame}


\begin{frame}{}
    \centering
    Viewed from a GUI (macOS \& Windows)\\
    {\bf GitHub Desktop} Walkthrough\par
    \vspace{3em}
    \url{https://desktop.github.com/}
\end{frame}

\imageframe[caption=Log in to your GitHub account,color=black]{github-windows/1}
\imageframe[caption=Create a (local) repository,color=black]{github-windows/2}
\imageframe[caption=GitHub Desktop has already made a first commit on your behalf,color=black]{github-windows/3}
\imageframe[caption=Open the repo in Windows Explorer,color=black]{github-windows/4}
\imageframe[caption=Add a simple README.md...,color=black]{github-windows/5}
\imageframe[caption=The change is listed in the Changes panel,color=black]{github-windows/6}
\imageframe[caption=Write a commit message \& commit!,color=black]{github-windows/7}
\imageframe[caption=The History panel shows the log and a diff of your changes,color=black]{github-windows/8}

\begin{frame}{}
    Viewed from a GUI\\
    {\bf Tortoise GIT}\par
    \vspace{3em}
    \url{https://tortoisegit.org/}
\end{frame}

\imageframe[color=black,caption=Direct interaction in the Windows explorer]{tortoise-git/gitcommit}
\imageframe[color=black,caption=Files' status appear as icons]{tortoise-git/overlays}
\imageframe[color=black,caption=All the functionalities are available]{tortoise-git/contextmenu}
\imageframe[color=black,caption=Commit window]{tortoise-git/commit}


\section{Collaborating}

\imageframe{dvcs-1}
\imageframe{dvcs-2}
\imageframe{dvcs-3}
\imageframe{dvcs-4}
\imageframe{dvcs-5}
\imageframe{dvcs-6}
\imageframe{dvcs-7}
\imageframe{dvcs-8}



\begin{frame}[fragile]{}

\centering

    \begin{tikzpicture}
    % Commit DAG
    \onslide<1-3>{
      \gitDAG[grow right sep = 2em]{
        A -- B -- C;
      };

      % Branch
      \gitbranch
        {master}     % node name and text 
        {above=of C} % node placement
        {C}          % target
    }



      % HEAD reference
    \onslide<1-2>{
      \gitHEAD
        {above=of master} % node placement
        {master}          % target
    }

    \onslide<3>{
      \gitremotebranch
        [origmaster]    % node name
        {origin/master} % node text
        {above=of master}    % node placement
        {master}             % target
      \gitHEAD
        {above=of origmaster} % node placement
        {origmaster}          % target
    }

    \end{tikzpicture}

\vspace{3em}
    \begin{overlayarea}{\textwidth}{5cm}
\only<2>{
\small
\texttt{git remote add origin git@github.com:user/repo.git}\par
\scriptsize
\texttt{git remote add john-usb E:\textbackslash john\_repo}\\
\texttt{git remote add ftp-origin ftp://host.xz/path/to/repo.git/}\\
...\\
}

\only<3>{
\Large
\texttt{git push origin master} \par
\normalsize
(or simply \texttt{git push})
}
\end{overlayarea}
\end{frame}

\begin{frame}[fragile]{}

    \centering

    \begin{tikzpicture}
        \onslide<1-2>{
            \gitDAG[grow right sep = 2em]{
                A -- B -- C -- D -- E;
            };

            % Branch
            \gitbranch
            {master}     % node name and text 
            {above=of E} % node placement
            {E}          % target

            % HEAD reference
            \gitHEAD
            {above=of master} % node placement
            {master}          % target

            \gitremotebranch
            [origmaster]    % node name
            {origin/master} % node text
            {above=of C}    % node placement
            {C}             % target
        }

        \onslide<3>{
            \gitDAG[grow right sep = 2em]{
                A -- B -- C -- {
                    F,
                    D -- E,
                }
            };

            % Branch
            \gitbranch
            {master}     % node name and text 
            {above=of E} % node placement
            {E}          % target

            % HEAD reference
            \gitHEAD
            {above=of master} % node placement
            {master}          % target

            \gitremotebranch
            [origmaster]    % node name
            {origin/master} % node text
            {above=of F}    % node placement
            {F}             % target
        }

        \onslide<4-5>{
            \gitDAG[grow right sep = 2em]{
                A -- B -- C -- {
                    F -- D' -- E',
                    {[nodes=unreachable] D -- E },
                }
            };

            % Branch
            \gitbranch
            {master}     % node name and text 
            {above=of E'} % node placement
            {E'}          % target

            % HEAD reference
            \gitHEAD
            {above=of master} % node placement
            {master}          % target

            \gitremotebranch
            [origmaster]    % node name
            {origin/master} % node text
            {above=of F}    % node placement
            {F}             % target
        }

        \onslide<6->{
            \gitDAG[grow right sep = 2em]{
                A -- B -- C -- {
                    F -- D' -- E',
                    {[nodes=unreachable] D -- E },
                }
            };

            % Branch
            \gitbranch
            {master}     % node name and text 
            {above=of E'} % node placement
            {E'}          % target


            \gitremotebranch
            [origmaster]    % node name
            {origin/master} % node text
            {above=of master}    % node placement
            {master}             % target

            \gitHEAD
            {above=of origmaster} % node placement
            {origmaster}          % target
        }


    \end{tikzpicture}

    \vspace{3em}
    \begin{overlayarea}{\textwidth}{5cm}
    \only<2>{
        What happened on our remote? Let's have a look...\\
        \Large
        \texttt{git fetch origin}\\
    }

    \only<4>{
        \Large
        \texttt{git rebase origin/master}\par
        \normalsize
        (but you don't need it, because...)\\
    }
    \only<5>{
        \Large
        \texttt{git pull --rebase}\\
    }
    \only<6>{
        \Large
        \texttt{git push}\\
    }
    \end{overlayarea}

\end{frame}



\begin{frame}[fragile]{To summarize...}

The first time...
\begin{shcode}
$ git clone <url>
# for instance,
# git clone https://github.com/user/repo.git
\end{shcode}
Then...
\begin{shcode}
$ cd <repo>
# make some changes...
$ git add <files>
$ git commit -m"<commit message>"
# ...
# when you want to share:
$ git pull --rebase # any changes on the remote?
$ git push
\end{shcode}

\end{frame}

\section{The dreadful conflicts}

\begin{frame}[fragile]{The dreadful conflict}

    While peacefully editing your last (great) paper...

\begin{shcode}
$ git pull --rebase john master
First, rewinding head to replay your work on top of it...
Applying: Better terminology
Using index info to reconstruct a base tree...
M	main.tex
Falling back to patching base and 3-way merge...
Auto-merging main.tex
CONFLICT (content): Merge conflict in main.tex
error: Failed to merge in the changes.
Patch failed at 0001 Better terminology
The copy of the patch that failed is found in: .git/rebase-apply/patch

When you have resolved this problem, run "git rebase --continue".
If you prefer to skip this patch, run "git rebase --skip" instead.
To check out the original branch and stop rebasing, run "git rebase --abort".
\end{shcode}
\end{frame}

\begin{frame}[fragile]{}
\begin{shcode}
$ git pull --rebase john master
# conflict!
$ git mergetool
\end{shcode}
\end{frame}

\imageframe[color=black,caption=Meld is one of the nice tools to fix conflicts]{meld}

\section{Social coding: GitHub workflow}

\imageframe[caption=GitHub,color=black]{github-morse}
\imageframe[caption=BitBucket,color=black]{bitbucket}
\imageframe[caption=GitLab -- open-source You can install it on your own server,color=black]{gitlab}

\imageframe[color=black]{github-morse}

\imageframe{github-workflow-1}
\imageframe{github-workflow-2}
\imageframe{github-workflow-3}
\imageframe{github-workflow-4}
\imageframe{github-workflow-5}
\imageframe{github-workflow-6}


\begin{frame}[fragile]{What happened exactly?}

    \vspace{1em}
    \centering

    \begin{multicols}{2}
        \resizebox{\columnwidth}{!}{%
    \begin{tikzpicture}
        \onslide<1>{
            \gitDAG[grow right sep = 2em]{
                A -- B -- C -- D -- E;
            };

            % Branch
            \gitbranch
            {master}     % node name and text 
            {above=of E} % node placement
            {E}          % target

            % HEAD reference
            \gitHEAD
            {above=of master} % node placement
            {master}          % target

            \gitremotebranch
            [origmaster]    % node name
            {origin/master} % node text
            {above=of C}    % node placement
            {C}             % target
        }

        \onslide<2>{
            \gitDAG[grow right sep = 2em]{
                A -- B -- C -- {
                    F,
                    D -- E,
                }
            };

            % Branch
            \gitbranch
            {master}     % node name and text 
            {above=of E} % node placement
            {E}          % target

            \gitremotebranch
            [johnmaster]    % node name
            {john/master} % node text
            {above=1.2 of F}    % node placement
            {F}             % target


            \gitremotebranch
            [origmaster]    % node name
            {origin/master} % node text
            {above=of C}    % node placement
            {C}             % target

            \gitHEAD
            {above=of master} % node placement
            {master}          % target

        }

        \onslide<3>{
            \gitDAG[grow right sep = 2em]{
                A -- B -- C -- {
                    F -- D' -- E',
                    {[nodes=unreachable] D -- E },
                }
            };

            % Branch
            \gitbranch
            {master}     % node name and text 
            {above=of E'} % node placement
            {E'}          % target

            \gitremotebranch
            [origmaster]    % node name
            {origin/master} % node text
            {above=of C}    % node placement
            {C}             % target

            \gitremotebranch
            [johnmaster]    % node name
            {john/master} % node text
            {above=1.2 of F}    % node placement
            {F}             % target


            \gitHEAD
            {above=of master} % node placement
            {master}          % target


        }
        \onslide<4>{
            \gitDAG[grow right sep = 2em]{
                A -- B -- C -- {
                    F -- D' -- E',
                    {[nodes=unreachable] D -- E },
                }
            };

            % Branch
            \gitbranch
            {master}     % node name and text 
            {above=of E'} % node placement
            {E'}          % target

            \gitremotebranch
            [origmaster]    % node name
            {origin/master} % node text
            {above=of master}    % node placement
            {master}             % target

            \gitremotebranch
            [johnmaster]    % node name
            {john/master} % node text
            {above=1.2 of F}    % node placement
            {F}             % target


            \gitHEAD
            {above=of origmaster} % node placement
            {origmaster}          % target


        }

    \end{tikzpicture}
    }

    \only<1>{
        \includegraphics[width=0.8\columnwidth]{github-workflow-details-3}
    }
    \only<2-3>{
        \includegraphics[width=0.8\columnwidth]{github-workflow-details-5}
    }

    \only<4>{
        \includegraphics[width=0.8\columnwidth]{github-workflow-details-6}
    }

    \end{multicols}

    \begin{overlayarea}{\textwidth}{5cm}
    \only<1>{
        After forking on GitHub, Paul runs\\
        \texttt{git clone https://github.com/paul/cool\_app.git}\\
        \normalsize
        and he adds few local commits\\
    }

    \only<2>{
        He would like to propose his changes to John\\
        First, he needs to get the latest changes from John:\par
        \footnotesize
        \texttt{git add remote john https://github.com/john/cool\_app.git}\\
        \normalsize
        \texttt{git fetch john}\\
    }

    \only<3>{
        Paul rebases his \texttt{master} branch on John's one:\par
        \texttt{git rebase john/master}\par
        \footnotesize
        (actually, Paul would simply run \texttt{git pull --rebase john master})
    }
    \only<4>{
        He pushes his commits to his own GitHub account:\par
        \texttt{git push}\\
        ...and finally press the ``Create a pull request'' button in GitHub.
    }

\end{overlayarea}
\end{frame}

\begin{frame}{}
    \centering
    (what happens next on John's side is a story for another day :-) But to make
    it short, he can press ``Merge pull request'' on his GitHub account if he is
    happy with the pull-request!)
\end{frame}
\imageframe{github-workflow-7}


\section{The one slide to remember}

\begin{frame}{GIT cheat sheet}
\scriptsize
    \begin{multicols}{2}

    {\bf To start...}\\
    ...from scratch: \texttt{git init}\\
    ...from existing repo: \texttt{git clone <url>}\par

    \rule{\columnwidth}{0.2pt}

    {\bf Prepare commits:}\\
    \texttt{git add}\\
    \texttt{git rm}\\
    \texttt{git add -p} (partial files)\par

    {\bf Commit:}\\
    \texttt{git commit}\par

    \rule{\columnwidth}{0.2pt}

    {\bf Create branch:}\\
    \texttt{git checkout -b <branch>}\par

    {\bf Jump between branches:}\\
    \texttt{git checkout <branch>}\par

    {\bf ``Import'' another branch:}\\
    \texttt{git rebase <other\_branch>}\par

    \rule{\columnwidth}{0.2pt}

    {\bf Add a remote source:}\\
    \texttt{git remote add <name> <url>}\par

    {\bf What's new on a remote?}\\
    \texttt{git pull <remote> <branch>}\\
    {\tiny (\texttt{git pull} alone $\equiv$ \texttt{git pull origin master})}\par

    {\bf Share stuff on a remote:}\\
    \texttt{git push <remote> <branch>}\\
    {\tiny (\texttt{git push} alone $\equiv$ \texttt{git push origin master})}\par

    \rule{\columnwidth}{0.2pt}

    \begin{multicols}{2}
    {\bf Repo state}\\
    \texttt{git status}\par

    {\bf Repo history}\\
    \texttt{git log}\par

    {\bf Who did what?}\\
    \texttt{git blame}\par

    {\bf I've lost everythg!}\\
    \texttt{git reflog}\par


    \end{multicols}

    ~\\

    \end{multicols}

\end{frame}

\imageframe[caption=That's all folks! The slides are on-line: https://github.com/severin-lemaignan/git-presentation]{git}

\section{Working with branches}

\begin{frame}{Branches}

    \centering

    \begin{tikzpicture}
        \only<1-5>{
        \onslide<1>{
            \gitDAG[grow right sep = 2em]{
                A -- B -- C;
            };

            % Branch
            \gitbranch
            {master}     % node name and text 
            {above=of C} % node placement
            {C}          % target

            % HEAD reference
            \gitHEAD
            {above=of master} % node placement
            {master}          % target

        }

        \onslide<2>{
            \gitDAG[grow right sep = 2em]{
                A -- B -- C;
            };

            % Branch
            \gitbranch
            {master}     % node name and text 
            {above=of C} % node placement
            {C}          % target

            \gitbranch
            {cool-idea}     % node name and text 
            {above=of master} % node placement
            {master}          % target

            % HEAD reference
            \gitHEAD
            {above=of cool-idea} % node placement
            {cool-idea}          % target
        }

        \onslide<3>{
            \gitDAG[grow right sep = 2em]{
                A -- B -- C -- D -- E;
            };

            % Branch
            \gitbranch
            {master}     % node name and text 
            {above=of C} % node placement
            {C}          % target

            \gitbranch
            {cool-idea}     % node name and text 
            {above=of E} % node placement
            {E}          % target

            % HEAD reference
            \gitHEAD
            {above=of cool-idea} % node placement
            {cool-idea}          % target
        }

        \onslide<4>{
            \gitDAG[grow right sep = 2em]{
                A -- B -- C -- D -- E;
            };

            % Branch
            \gitbranch
            {master}     % node name and text 
            {above=of C} % node placement
            {C}          % target

            \gitbranch
            {cool-idea}     % node name and text 
            {above=of E} % node placement
            {E}          % target

            % HEAD reference
            \gitHEAD
            {above=of master} % node placement
            {master}          % target
        }

        \onslide<5>{
            \gitDAG[grow right sep = 2em]{
                A -- B -- C -- {
                    D -- E,
                    F -- G -- H,
                }
            };

            % Branch
            \gitbranch
            {master}     % node name and text 
            {above=of H} % node placement
            {H}          % target

            \gitbranch
            {cool-idea}     % node name and text 
            {above=of E} % node placement
            {E}          % target

            % HEAD reference
            \gitHEAD
            {above=of master} % node placement
            {master}          % target
        }
    }
        \only<6->{
            \gitDAG[grow right sep = 1.5em]{
                A -- B -- C -- {
                    D -- E,
                    F -- G -- H -- {
                        K,
                        I -- J
                        },
                }
            };

            % Branch
            \gitbranch
            {master}     % node name and text 
            {above=of K} % node placement
            {K}          % target

            \gitbranch
            {cool-idea}     % node name and text 
            {above=of E} % node placement
            {E}          % target

            \gitbranch
            {bug142}     % node name and text 
            {above=of J} % node placement
            {J}          % target


            % HEAD reference
            \gitHEAD
            {above=of master} % node placement
            {master}          % target
        }


    \end{tikzpicture}

    \vspace{3em}
    \centering
    \only<1>{
        What if...?\\
    }
    \only<2>{
        \Large
        \texttt{git checkout -b cool-idea}\\
    }
    \only<5>{
        The branch name is an alias for the tip of the current branch\\
    }
    \only<6>{
        $\Rightarrow$ branches are very cheap\\ 
        +10 of them at a given time it not uncommon\\
    }
    \uncover<4>{
        Let go back to serious stuff!\\
        \Large
        \texttt{git checkout master}
    }


\end{frame}


\begin{frame}{Merging branches}

    \centering

\begin{tikzpicture}[
    >=latex,
    every edge/.style={draw,thick,hriSec3}
]

        \onslide<1>{
            \gitDAG[grow right sep = 2em]{
                B -- C -- {
                    D -- E,
                    F -- G -- H,
                };
            };

            % Branch
            \gitbranch
            {master}     % node name and text 
            {above=of H} % node placement
            {H}          % target

            \gitbranch
            {cool-idea}     % node name and text 
            {above=of E} % node placement
            {E}          % target

            % HEAD reference
            \gitHEAD
            {above=of master} % node placement
            {master}          % target

            \node[DAGcommit,right=3 of E.south,dashed] (merge) {?};
            \path[draw] (E) edge [->, bend left] (merge);
            \path[draw] (H) edge [->, bend right] (merge);
    }
        \onslide<2>{
            \gitDAG[grow right sep = 2em]{
                B -- C -- {
                    D -- E,
                    F -- G -- H -- merge,
                };

                E -- merge;
            };

            % Branch
            \gitbranch
            {master}     % node name and text 
            {above=of merge} % node placement
            {merge}          % target

            \gitbranch
            {cool-idea}     % node name and text 
            {above=of E} % node placement
            {E}          % target

            % HEAD reference
            \gitHEAD
            {above=of master} % node placement
            {master}          % target

    }
        \onslide<3>{
            \gitDAG[grow right sep = 2em]{
                B -- C -- {
                    D -- E,
                    F -- G -- H -- I -- J,
                };

                E -- I;
            };

            % Branch
            \gitbranch
            {master}     % node name and text 
            {above=of J} % node placement
            {J}          % target

            \gitbranch
            {cool-idea}     % node name and text 
            {above=of E} % node placement
            {E}          % target

            % HEAD reference
            \gitHEAD
            {above=of master} % node placement
            {master}          % target

    }
        \onslide<4>{
            \gitDAG[grow right sep = 2em]{
                B -- C -- {
                    D -- E -- K -- L,
                    F -- G -- H -- I -- J,
                };

                E -- I;
            };

            % Branch
            \gitbranch
            {master}     % node name and text 
            {above=of J} % node placement
            {J}          % target

            \gitbranch
            {cool-idea}     % node name and text 
            {above=of L} % node placement
            {L}          % target

            % HEAD reference
            \gitHEAD
            {above=of cool-idea} % node placement
            {cool-idea}          % target

    }



    \end{tikzpicture}

    \vspace{3em}
    \centering
    \only<1>{
        Two options: {\bf merging} and {\bf rebasing}\\
    }
    \only<2>{
        Merging\\
        \Large
        \texttt{git merge cool-idea}\\
    }
    \only<3>{
        \Large
        \texttt{git commit}\\
    }
    \uncover<4>{
        \Large
        \texttt{git checkout cool-idea}\\
        \texttt{git commit}\\
        \normalsize
        ...etc.
    }


\end{frame}

\begin{frame}{Rebasing branches}

    \centering

\begin{tikzpicture}[
    >=latex,
    every edge/.style={draw,thick,hriSec3}
]

        \onslide<1>{
            \gitDAG[grow right sep = 2em]{
                B -- C -- {
                    D -- E,
                    F -- G -- H,
                };
            };

            % Branch
            \gitbranch
            {master}     % node name and text 
            {above=of H} % node placement
            {H}          % target

            \gitbranch
            {cool-idea}     % node name and text 
            {above=of E} % node placement
            {E}          % target

            % HEAD reference
            \gitHEAD
            {above=of master} % node placement
            {master}          % target

            \node[DAGcommit,right=3 of E.south,dashed] (merge) {?};
            \path[draw] (E) edge [->, bend left] (merge);
            \path[draw] (H) edge [->, bend right] (merge);
    }
        \onslide<2>{
            \gitDAG[grow right sep = 2em]{
                B -- C -- {
                    D -- E -- F' -- G' -- H',
                    {[nodes=unreachable] F -- G -- H },
                };
            };

            % Branch
            \gitbranch
            {master}     % node name and text 
            {above=of H'} % node placement
            {H'}          % target

            \gitbranch
            {cool-idea}     % node name and text 
            {above=of E} % node placement
            {E}          % target

            % HEAD reference
            \gitHEAD
            {above=of master} % node placement
            {master}          % target

    }
        \onslide<3>{
            \gitDAG[grow right sep = 2em]{
                B -- C -- {
                    D -- E -- {
                        I -- J,
                        F' -- G' -- H',
                    },
                    {[nodes=unreachable] F -- G -- H },
                };
            };

            % Branch
            \gitbranch
            {master}     % node name and text 
            {above=of H'} % node placement
            {H'}          % target

            \gitbranch
            {cool-idea}     % node name and text 
            {above=of J} % node placement
            {J}          % target

            % HEAD reference
            \gitHEAD
            {above=of cool-idea} % node placement
            {cool-idea}          % target

    }

    \end{tikzpicture}

    \vspace{3em}
    \centering
    \only<2>{
        Rebasing\\
        \Large
        \texttt{git rebase cool-idea}\\
    }
    \uncover<3>{
        \Large
        \texttt{git checkout cool-idea}\\
        \texttt{git commit}\\
    }


\end{frame}

\begin{frame}{More commit aliases: Tags}

    \centering

    \begin{tikzpicture}
            \gitDAG[grow right sep = 1.5em]{
                A -- B -- C -- {
                    D -- E,
                    F -- G -- H -- {
                        K,
                        I -- J
                        },
                }
            };

            \gittag
            [v12]
            {v1.2}
            {above=of C}
            {C}

            \gitbranch
            {master}
            {above=of K}
            {K}

            \gitbranch
            {cool-idea}
            {above=of E}
            {E}

            \gitbranch
            {bug142}
            {above=of J}
            {J}

            \gittag
            {WP5}
            {above=of H}
            {H}


            % HEAD reference
            \gitHEAD
            {above=of master} % node placement
            {master}          % target


    \end{tikzpicture}

    \vspace{3em}
    \centering
        {\bf Label} important commits/milestones\\
        \Large
        \texttt{git tag v1.2}\\
        \texttt{git tag WP5}


\end{frame}



\begin{frame}[fragile]{To summarize...}

\begin{shcode}
# where are we?
$ git branch
master
# make some changes...
$ git add <files> && git commit -m"<commit message>"
# start working on something new?
$ git checkout -b new-idea
$ git branch
new-idea
# work in that branch for a while
$ git add <files> && git commit -m"<commit message>"
# back to master
$ git checkout master
#...
# rebase master on new-idea: new-idea is now in master
$ git rebase new-idea
\end{shcode}

\end{frame}

\begin{frame}{}
    \centering
    {Viewed from a GUI...}
\end{frame}

\imageframe[caption=We can easily create a new branch,color=black]{github-windows/9}
\imageframe[caption=We can compare numerical\_coordinates with master (click on View branch for the full history),color=black]{github-windows/10}
\imageframe[caption=We can jump between branches...,color=black]{github-windows/11}
\imageframe[caption=...and watch how they diverge,color=black]{github-windows/12}
\imageframe[caption=We switch back to numerical\_coordinates and merge in master,color=black]{github-windows/13}
\imageframe[caption=The merge commit is reflected in the history of the branch,color=black]{github-windows/14}


\section{Etiquette of social coding 101}

\begin{frame}{}
    \centering

    \highlight{\bf principle of least surprise}

    Make people feel at home when they interact with your project!

\end{frame}

\begin{frame}{}
    \centering
    \highlight{one repo = one thing}

    make plenty of repos!
\end{frame}

\begin{frame}[fragile]{Repository layout}

Try to follow as much as possible the {\bf Filesystem Hierarchy
 Standard} (FHS). Mainly:

\begin{shcode}
src/        # source
include/    # *public* headers
etc/        # configuration files
share/      # data
doc/        # documentation
README
LICENSE
\end{shcode}

\centering

\only<1>{
{\bf NO build artifacts!!}\\
{\bf no binaries} (except possibly in \sh{share/})
}
\only<2>{
\sh{README} (or better, use markdown: \sh{README.md}): what is
the project about? who is the target audience? how to install? how to get started?
}

\end{frame}

\begin{frame}{License}
    \begin{itemize}
        \item {\bf no license} $\Rightarrow$ default copyright laws apply.
            You (or probably UoP) retain all rights to your source code; nobody else may
            reproduce, distribute, or create derivative works from your work.
        \item {\bf Permissive licenses}: others do essentially whatever they
            want with your code, as long as they give your attribution.
            Examples: MIT, BSD
        \item {\bf Copyleft licenses}: Derivative work must be made
            available under the same terms as the original work (\emph{viral
            licenses}). Example: GPL
    \end{itemize}

    \centering
    \only<1>{
        {\bf You always keep the author rights!}\\
        $\Rightarrow$ you can change the
        license at any time.
    }

    \only<2>{
    Check \url{http://choosealicense.com/}\\
    and discuss that with your supervisor
    }


\end{frame}

\begin{frame}{Build system}

    Use and provide a build system!

    \begin{itemize}
        \item Windows-only $\Rightarrow$ a Visual Studio solution is ok
        \item MacOS-only $\Rightarrow$ a XCode project is ok
    \end{itemize}

    In all other cases, go for a cross-platform build system like {\bf
    CMake}.
\end{frame}

\begin{frame}{Commit hygiene}
    \centering

    {\bf ``Show me the project history, I'll tell you what coder you are''}

    \begin{itemize}
        \item<1> {\bf Commit often!} Push when needed (or at the end of day)
        \item<2> Write useful messages (no ``\texttt{Fixed bug}'' or ``\texttt{New
            file}'')
        \item<2> First line of commit messages < 72 characters
        \item<3> Tag important commits!
    \end{itemize}

    \only<1>{
        Because commits are local (ie, private), {\bf do commit often}:
        {\bf mistakes are ok} as you can fix them before sharing with others.
    }
    \only<3>{
        Notably, GitHub (amongst others) interpret tags as {\bf releases} of
        your code.
    }
\end{frame}

\begin{frame}{A few cool GitHub stuff to finish}
    Besides bugtracking, project homepages and wikis,GitHub integrates with many
    third-party services \& tools:

    \begin{itemize}
        \item {\bf Travis CI} or {\bf AppVeyor} for continuous integration
    \end{itemize}
\end{frame}

\imageframe{pr-failed-ci}

\begin{frame}{A few cool stuff to finish}
    + GitHub integrates with many external services 
    \& tools:

    \begin{itemize}
        \item {\bf Travis CI} or {\bf AppVeyor} for continuous integration
        \item {\bf zenodo}: associate a DOI to your repository
        \item {\bf ReadTheDocs}: generate and publish on-line 
            documentation
    \end{itemize}
\end{frame}

\end{document}






